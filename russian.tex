%% start of file `template.tex'.
%% Copyright 2006-2013 Xavier Danaux (xdanaux@gmail.com).
%
% This work may be distributed and/or modified under the
% conditions of the LaTeX Project Public License version 1.3c,
% available at http://www.latex-project.org/lppl/.


\documentclass[11pt,a4paper,sans]{moderncv}        % possible options include font size ('10pt', '11pt' and '12pt'), paper size ('a4paper', 'letterpaper', 'a5paper', 'legalpaper', 'executivepaper' and 'landscape') and font family ('sans' and 'roman')

\moderncvstyle{banking}                            % style options are 'casual' (default), 'classic', 'oldstyle' and 'banking'
\moderncvcolor{blue}                                % color options 'blue' (default), 'orange', 'green', 'red', 'purple', 'grey' and 'black'


\usepackage{polyglossia}   %% загружает пакет многоязыковой вёрстки

\usepackage{fontspec}
\usepackage{xunicode}
\usepackage{xltxtra}
\usepackage{amssymb}
\defaultfontfeatures{Scale=MatchLowercase,Mapping=tex-text}

\setdefaultlanguage[spelling=modern]{russian}  %% устанавливает главный язык документа
\setotherlanguage{english} %% объявляет второй язык документа

\newfontfamily\cyrillicfonttt{Liberation Sans}
\newfontfamily\cyrillicfont{Liberation Sans}



% adjust the page margins
\usepackage[scale=0.75]{geometry}

\name{Алексей}{Уйманов}
\address{Россия}{Тюмень}{}
\phone[mobile]{+7 932-470-79-15}                   % optional, remove / comment the line if not wanted
\email{s9gf4ult@gmail.com}                               % optional, remove / comment the line if not wanted
\jabber{segfault@jabber.ru}
\homepage{https://github.com/s9gf4ult}
%% \quote{Решатель проблем}

\newcommand*{\hlink}[1]{\textcolor{blue}{\texttt{\underline{\href{#1}{#1}}}}}


\begin{document}
\makecvtitle

\section{Общее}
\cvitem{Возраст}{27}
\cvitem{Семейное положение}{Холост}

\section{Образование}
\cventry{2007 -- 2009}{Магистр}{ТюмГНГУ}{Тюмень}{\textit{4.8/5}}{Кафедра прикладной механики}

\cventry{2003 -- 2007}{Бакалавр}{ТюмГНГУ}{Тюмень}{\textit{4.7/5}}{Кафедра прикладной механики}

\section{Опыт работы}
\subsection{Профессиональный}

\cventry{2013 (текущее место работы)}{Программист}{CyberStep}{Тюмень}{}
        {Разработка встраиваемого программного обеспечения для станков плазменной резки.}

\cventry{2012 -- 2013}{Веб програмист}{Ирис Интеграция}{Тюмень}{}{%
  Разработка сайтов на PHP, Ruby on Rails, Python + Django
  \begin{itemize}
  \item \hlink{http://qrgorod.com}: Разработка архитектуры и программирование, верстка \hbox{(PHP + Mediawiki)}
  \item \hlink{http://qrgolos.com}: Программирование бэкэнда на Ruby on Rails
  \end{itemize}}

\cventry{2009 -- 2012}{Системный администратор}{ГАУ ТО Миац}{Тюмень}{}{%
  Поддержка систем хранения, виртуальных машин.}

\cventry{2007 -- 2009}{Инженер конструктор}{ОАО Сибнефтемаш}{Тюмень}{}{%
  Прочностной расчет технологических емкостей, конструкций}

\cventry{2006 -- 2007}{Системный администратор}{МУПГТ Тюменьгортранс}{Тюмень}{}{}

\subsection{Мои проекты}
\cventry {2013}{Haskell}{HDBI}{\hlink{https://github.com/s9gf4ult/hdbi}}{}{%
  Форк оригинального HDBC -- обобщенного интерфейса для работы с базами данных.}
    
\cventry {2013}{Haskell}{HDBI-postgresql}{\hlink{https://github.com/s9gf4ult/hdbi-postgresql}}{}{%
  Драйвер, реализующий работу с PostgreSQL. Сделан для выше указанного HDBI.}

\cventry {2013}{Haskell}{HDBI-sqlite}{\hlink{https://github.com/s9gf4ult/hdbi-sqlite}}{}{%
  Драйвер, реализующий работу с Sqlite при помощи HDBI.}

\cventry {2013}{Haskell}{quickcheck-assertions}%
    {\hlink{https://github.com/s9gf4ult/quickcheck-assertions}}{}{%
      Дополнительный пакет для QuickCheck -- пакета для проверки свойств программы и тестирования.}
    
\cventry{2011--2012}{Python}{Track Deal}%
        {\hlink{https://github.com/s9gf4ult/track-deal}}{}{%
          Утилита для ведения журнала трейдера.}

\cvitem{Прочее}{Участие в различных OpenSource проектах, отправка патчей с
  исправлением ошибок}

\section{Умения}
  
\cvitem{Программирование}{%
  \begin{itemize}
  \item Haskell -- опыт около года
  \item Python -- опыт более двух лет, PyGtk, Django
  \item C -- опыт пол года
  \item Sql -- PostgreSQL, MySql, Sqlite, Oracle
  \item Git, SVN, Hg
  \item Bash -- создание скриптов автоматизации
  \item NoSql -- MongoDB, memcached
  \item Ruby -- опыт разрабтки двух приложений на Ruby on Rails (Rspec, Factory girl, Faye, Devise)
  \item HTML/CSS -- большой опыт верстки
  \item JavaScript -- JQuery, AngularJS
  \item PHP -- опыт работы с движком Mediawiki около года
  \end{itemize}}

\cvitem{Администрирование}{%
  \begin{itemize}
  \item Linux -- Опытный пользователь и администратор Linux.  На домашнем
    компьютере Linux более \hbox{4 лет}
  \item Apache, nginx -- конфигурирование и установка модулей
  \item Varnish -- написание скрптов кеширования на VCL
  \item Squid -- конфигурирование
  \item Хранение -- LVM, RAID, iSCSI
  \item Сеть -- iptables, ip, tc
  \item Виртуализация -- libvirt, kvm
  \end{itemize}}

\cvitem{Прочее}{%
  \begin{itemize}
  \item \LaTeX
  \item Emacs
  \end{itemize}}
  

\section{Иностранные языки}
\cvitemwithcomment{Английский}{}{Свободное чтение документации и электронная переписка}

\section{О себе}
Считаю корректность, правильность и протестированность кода важнее скорости его
работы, всегда стараюсь придерживаться TDD. Из систем контроля версий
предпочитаю Git. Из языков программирования предпочитаю Haskell.

\nocite{*}
\bibliographystyle{plain}
\bibliography{publications}                        % 'publications' is the name of a BibTeX file
\end{document}
