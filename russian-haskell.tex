%% start of file `template.tex'.
%% Copyright 2006-2013 Xavier Danaux (xdanaux@gmail.com).
%
% This work may be distributed and/or modified under the
% conditions of the LaTeX Project Public License vergqsion 1.3c,
% available at http://www.latex-project.org/lppl/.


\documentclass[11pt,a4paper,sans]{moderncv}        % possible options include font size ('10pt', '11pt' and '12pt'), paper size ('a4paper', 'letterpaper', 'a5paper', 'legalpaper', 'executivepaper' and 'landscape') and font family ('sans' and 'roman')

\moderncvstyle{banking}                            % style options are 'casual' (default), 'classic', 'oldstyle' and 'banking'
\moderncvcolor{blue}                                % color options 'blue' (default), 'orange', 'green', 'red', 'purple', 'grey' and 'black'


\usepackage{polyglossia}   %% загружает пакет многоязыковой вёрстки

\usepackage{fontspec}
\usepackage{xunicode}
\usepackage{xltxtra}
\usepackage{amssymb}
\defaultfontfeatures{Scale=MatchLowercase,Mapping=tex-text}

\setdefaultlanguage[spelling=modern]{russian}  %% устанавливает главный язык документа
\setotherlanguage{english} %% объявляет второй язык документа

\newfontfamily\cyrillicfonttt{Liberation Sans}
\newfontfamily\cyrillicfont{Liberation Sans}



% adjust the page margins
\usepackage[scale=0.77]{geometry}

\name{Алексей}{Уйманов}
\address{Россия}{Тюмень}{}
\phone[mobile]{+7 932-470-79-15}                   % optional, remove / comment the line if not wanted
\email{s9gf4ult@gmail.com}                               % optional, remove / comment the line if not wanted
\jabber{segfault@jabber.ru}
\homepage{https://github.com/s9gf4ult}
%% \homepage{https://bitbucket.org/s9gf4ult}
%% \quote{Решатель проблем}

\newcommand*{\hlink}[1]{\textcolor{blue}{\texttt{\underline{\href{#1}{#1}}}}}
\newcommand*{\nlink}[2]{\textcolor{blue}{\texttt{\underline{\href{#1}{#2}}}}}


\begin{document}
\makecvtitle

\section{Опыт работы}

\cventry{2014 -- Сейчас}{Haskell программист}{AdTel}{Удаленная работа}{}{}{%
  \begin{itemize}
  \item Сделал сервис аутентификации по протоколу Oauth2. Сервис в
    последствии был разбит на два пакета: подключаемый сервер-плагин
    oauth2-server (Yesod subsite) и непосредственно сервис, к которому
    подключен oauth2-server.
  \item Сделал пакет для удобной (и безопасной) генерации SQL-запросов
    \nlink{http://hackage.haskell.org/package/postgresql-query}{postgresql-query}. По
    сути, интерполирующий квазиквотер SQL-запросов с зачатками ORM.
  \item Переписал пакет OTP с использвоанием cryptohash:
    \nlink{http://hackage.haskell.org/package/one-time-password}{one-time-password}
  \item Сделал пакет для удобного доступа к redis соединению:
    \nlink{http://hackage.haskell.org/package/hedis-monadic}{hedis-monadic}
  \end{itemize}
}

\cventry{2014}{Ruby on Rails программист}{Студия Кечинова}{Удаленная работа}{}{}{%
  \begin{itemize}
  \item Закрыл столько-то задач в трекере ...
  \end{itemize}
}

\cventry{2013 -- 2014}{Программист}{CyberStep}{Тюмень}{}{%
  Разработка программного обеспечения для станков плазменной резки.
  \medskip
  \begin{itemize}
  \item Интерфейс станка плазменной резки (Python, Tcl, C)
    \begin{itemize}
    \item Внедрил базу данных параметров резки
    \item Сделал модуль интерполяции параметров резки для режимов резки под углом
    \item Всякая текучка с интерфейсом (Python, Tcl)
    \end{itemize}
  \item Форк Fat Free CRM, добавил генератор комерческих предложений для нужд предприятия\\
    \hlink{https://github.com/s9gf4ult/fat\_free\_crm/tree/commoffers}
  \end{itemize}
}

\cventry{2012 -- 2013}{Веб програмист}{Ирис Интеграция}{Тюмень}{}{%
  Разработка сайтов на PHP, Ruby on Rails, Python + Django
}

\cventry{2009 -- 2012}{Системный администратор}{ГАУ ТО Миац}{Тюмень}{}{%
  Поддержка систем хранения, виртуальных машин.
}

\cventry{2007 -- 2009}{Инженер конструктор}{ОАО Сибнефтемаш}{Тюмень}{}{%
  Прочностной расчет технологических емкостей, конструкций
}

\cventry{2006 -- 2007}{Системный администратор}{МУПГТ Тюменьгортранс}{Тюмень}{}{}

\newpage

\section{Умения}

\subsection{Программирование}

\begin{itemize}
\item Haskell
  \begin{itemize}
  \item Template Haskell/QuasiQuoter: могу на Template Haskell
    делать всякое. Освоил при написании
    \nlink{http://hackage.haskell.org/package/postgresql-query}{postgresql-query}
  \item Комбинаторные парсеры: могу на parserc/attoparsec делать
    всякое, в частности, QQ-парсер в \hbox{postgresql-query} написан на
    attoparsec.
  \item Yesod: освоил фремворк за время работы в AdTel, делал
    пулл-реквесты, знаю внутреннее строение на хорошем уровне. Немного
    знаю Scotty
  \item QuicCheck/SmallCheck: автоматизированное тестирование
    свойств программы, стараюсь тестировать свойства программы по
    возможности.
  \item MonadBaseControl: знаю зачем нужно и могу написать инстанс
  \item Продвинутая магия типов: GADTs, TypeFamilies, RankNTypes
  \end{itemize}
\item Web
  \begin{itemize}
  \item CSS: Bootstrap, Blueprint CSS
  \item JavaScript: JQuery, AngularJS
  \end{itemize}
\item Python
  \begin{itemize}
  \item PyGtk
  \item Django (устаревшие знания)
  \end{itemize}
%% \item Ruby on Rails
%%   \begin{itemize}
%%   \item Разрабатывал

%% \item Общее
%%   \begin{itemize}
%%   \item TDD/BDD

%% \item Web:
%%   \begin{itemize}
%%   \item Ruby
%%     \begin{itemize}
%%     \item Ruby on Rails 3
%%     \item haml, coffeescript, scss
%%     \item rspec, factory\_girl, Faker
%%     \item paperclip, simple\_form, devise, acts\_as\_commentable
%%     \item ActiveAdmin, ransack, kaminari, formtastic
%%     \end{itemize}
%%   \item HTML/CSS: Blueprint CSS, немного Bootstrap
%%   \item JavaScript:  JQuery, AngularJS
%%   \item HTTP: REST
%%   \item Python: PyGtk, Django
%%   \item PHP: опыт работы с движком Mediawiki около года
%%   \end{itemize}
%% \item Sql: PostgreSQL, MySql, Sqlite
%% \item NoSql: MongoDB, memcached
%% \item Git
%% \item C
%% \item Haskell: GADTs, TypeFamilies, QuickCheck, Yesod, Conduit
\end{itemize}

\subsection{Администрирование}

\begin{itemize}
\item Linux
\item Apache, nginx
\item Varnish: Кеширование с VCL
\item Squid
\item Хранение: LVM, RAID, iSCSI
\item Сеть: iptables, ip, tc
\item Виртуализация: libvirt, kvm
\end{itemize}

\section{Образование}
\cventry{2007 -- 2009}{Магистр}{ТюмГНГУ}{Тюмень}{\textit{4.8/5}}{Кафедра прикладной механики}

\cventry{2003 -- 2007}{Бакалавр}{ТюмГНГУ}{Тюмень}{\textit{4.7/5}}{Кафедра прикладной механики}



\section{Прочее}

\begin{itemize}
\item \LaTeX
\item Emacs
\end{itemize}

\section{Иностранные языки}
\cvitemwithcomment{Английский}{}{Свободное чтение
  документации и электронная переписка}


\nocite{*}
\bibliographystyle{plain}
\bibliography{publications}                        % 'publications' is the name of a BibTeX file
\end{document}
