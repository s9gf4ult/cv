%% start of file `template.tex'.
%% Copyright 2006-2013 Xavier Danaux (xdanaux@gmail.com).
%
% This work may be distributed and/or modified under the
% conditions of the LaTeX Project Public License vergqsion 1.3c,
% available at http://www.latex-project.org/lppl/.


\documentclass[11pt,a4paper,sans]{moderncv}        % possible options include font size ('10pt', '11pt' and '12pt'), paper size ('a4paper', 'letterpaper', 'a5paper', 'legalpaper', 'executivepaper' and 'landscape') and font family ('sans' and 'roman')

\moderncvstyle{banking}                            % style options are 'casual' (default), 'classic', 'oldstyle' and 'banking'
\moderncvcolor{blue}                                % color options 'blue' (default), 'orange', 'green', 'red', 'purple', 'grey' and 'black'


\usepackage{polyglossia}   %% загружает пакет многоязыковой вёрстки

\usepackage{fontspec}
\usepackage{xunicode}
\usepackage{xltxtra}
\usepackage{amssymb}
\defaultfontfeatures{Scale=MatchLowercase,Mapping=tex-text}

\setdefaultlanguage[spelling=modern]{russian}  %% устанавливает главный язык документа
\setotherlanguage{english} %% объявляет второй язык документа

\newfontfamily\cyrillicfonttt{Liberation Sans}
\newfontfamily\cyrillicfont{Liberation Sans}



% adjust the page margins
\usepackage[scale=0.77]{geometry}

\name{Алексей}{Уйманов}
\address{Россия}{Тюмень}{}
\phone[mobile]{+7 932-470-79-15}                   % optional, remove / comment the line if not wanted
\email{s9gf4ult@gmail.com}                               % optional, remove / comment the line if not wanted
\jabber{segfault@jabber.ru}
\homepage{https://github.com/s9gf4ult}
\homepage{https://bitbucket.org/s9gf4ult}
%% \quote{Решатель проблем}

\newcommand*{\hlink}[1]{\textcolor{blue}{\texttt{\underline{\href{#1}{#1}}}}}


\begin{document}
\makecvtitle

\section{Образование}
\cventry{2007 -- 2009}{Магистр}{ТюмГНГУ}{Тюмень}{\textit{4.8/5}}{Кафедра прикладной механики}

\cventry{2003 -- 2007}{Бакалавр}{ТюмГНГУ}{Тюмень}{\textit{4.7/5}}{Кафедра прикладной механики}

\section{Опыт работы}
\subsection{Профессиональный}

\cventry{2014}{Ruby on Rails программист}{Удаленная работа}{}{}{}{%
  Имею небольшой опыт удаленной работы.
}

\cventry{2013 (текущее место работы)}{Программист}{CyberStep}{Тюмень}{}{%
  Разработка программного обеспечения для станков плазменной резки.
  \medskip
  \begin{itemize}
  \item Форк Fat Free CRM, добавил генератор комерческих предложений для фирмы\\
    \hlink{https://github.com/s9gf4ult/fat\_free\_crm/tree/commoffers}
  \end{itemize}
}

\cventry{2012 -- 2013}{Веб програмист}{Ирис Интеграция}{Тюмень}{}{%
  Разработка сайтов на PHP, Ruby on Rails, Python + Django
  \medskip
  \begin{itemize}
  \item \hlink{http://qrgorod.com}: Разработка архитектуры и программирование, верстка \hbox{(PHP + Mediawiki)}
  \item http://qrgolos.com (Домен уже свободен): Приложение для проведения
    массовых голосований в реальном времени с использованием QR-кодов (Бэкэнд).
  \end{itemize}}

\cventry{2009 -- 2012}{Системный администратор}{ГАУ ТО Миац}{Тюмень}{}{%
  Поддержка систем хранения, виртуальных машин.}

\cventry{2007 -- 2009}{Инженер конструктор}{ОАО Сибнефтемаш}{Тюмень}{}{%
  Прочностной расчет технологических емкостей, конструкций}

\cventry{2006 -- 2007}{Системный администратор}{МУПГТ Тюменьгортранс}{Тюмень}{}{}

\subsection{Мои проекты}

\cventry {2013}{Haskell}{HDBI}{\hlink{https://github.com/s9gf4ult/hdbi}}{}{%
  Форк оригинального HDBC -- обобщенного интерфейса для работы с базами
  данных. Также написаны драйверы для PostgreSQL и SQlite.}

\cventry {2013}{Haskell}{quickcheck-assertions}%
    {\hlink{https://github.com/s9gf4ult/quickcheck-assertions}}{}{%
      Дополнительный пакет для QuickCheck -- пакета для проверки свойств программы и тестирования.}

\cventry{2011--2012}{Python}{Track Deal}%
        {\hlink{https://github.com/s9gf4ult/track-deal}}{}{%
          Утилита для ведения журнала трейдера.}

\cvitem{Прочее}{Участие в различных OpenSource проектах, отправка патчей с
  исправлением ошибок}

\section{Умения}

\subsection{Программирование}

\begin{itemize}
\item Web:
  \begin{itemize}
  \item Ruby
    \begin{itemize}
    \item Ruby on Rails 3
    \item haml, coffeescript, scss
    \item rspec, factory\_girl, Faker
    \item paperclip, simple\_form, devise, acts\_as\_commentable
    \item ActiveAdmin, ransack, kaminari, formtastic
    \end{itemize}
  \item HTML/CSS: Blueprint CSS, немного Bootstrap
  \item JavaScript: JQuery, AngularJS
  \item HTTP: REST
  \item Python: PyGtk, Django
  \item PHP: опыт работы с движком Mediawiki около года
  \end{itemize}
\item Sql: PostgreSQL, MySql, Sqlite
\item NoSql: MongoDB, memcached
\item Git
\item C
\item Haskell: GADTs, TypeFamilies, QuickCheck, Yesod, Conduit
\end{itemize}

\subsection{Администрирование}

\begin{itemize}
\item Linux
\item Apache, nginx
\item Varnish: Кеширование с VCL
\item Squid
\item Хранение: LVM, RAID, iSCSI
\item Сеть: iptables, ip, tc
\item Виртуализация: libvirt, kvm
\end{itemize}

\subsection{Прочее}

\begin{itemize}
\item \LaTeX
\item Emacs
\end{itemize}

\section{Иностранные языки}
\cvitemwithcomment{Английский}{}{Свободное чтение
  документации и электронная переписка}


\nocite{*}
\bibliographystyle{plain}
\bibliography{publications}                        % 'publications' is the name of a BibTeX file
\end{document}
