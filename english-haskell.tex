%% start of file `template.tex'.
%% Copyright 2006-2013 Xavier Danaux (xdanaux@gmail.com).
%
% This work may be distributed and/or modified under the
% conditions of the LaTeX Project Public License version 1.3c,
% available at http://www.latex-project.org/lppl/.


\documentclass[11pt,a4paper,sans]{moderncv}        % possible options include font size ('10pt', '11pt' and '12pt'), paper size ('a4paper', 'letterpaper', 'a5paper', 'legalpaper', 'executivepaper' and 'landscape') and font family ('sans' and 'roman')

% moderncv themes
\moderncvstyle{banking}                            % style options are 'casual' (default), 'classic', 'oldstyle' and 'banking'
\moderncvcolor{blue}                                % color options 'blue' (default), 'orange', 'green', 'red', 'purple', 'grey' and 'black'
%\renewcommand{\familydefault}{\sfdefault}         % to set the default font; use '\sfdefault' for the default sans serif font, '\rmdefault' for the default roman one, or any tex font name
%\nopagenumbers{}                                  % uncomment to suppress automatic page numbering for CVs longer than one page

% character encoding
\usepackage[utf8]{inputenc}                       % if you are not using xelatex ou lualatex, replace by the encoding you are using
\usepackage{fullpage}
%% \usepackage{hyperref}
%\usepackage{CJKutf8}                              % if you need to use CJK to typeset your resume in Chinese, Japanese or Korean

% adjust the page margins
\usepackage[scale=0.76]{geometry}
%\setlength{\hintscolumnwidth}{3cm}                % if you want to change the width of the column with the dates
%\setlength{\makecvtitlenamewidth}{10cm}           % for the 'classic' style, if you want to force the width allocated to your name and avoid line breaks. be careful though, the length is normally calculated to avoid any overlap with your personal info; use this at your own typographical risks...

% personal data
\name{Aleksey}{Uymanov}
\address{}{Tyumen}{Russian Federation}% optional, remove / comment the line if not wanted; the "postcode city" and and "country" arguments can be omitted or provided empty
\phone[mobile]{+7 932-470-79-15}                   % optional, remove / comment the line if not wanted
\email{s9gf4ult@gmail.com}                               % optional, remove / comment the line if not wanted
\jabber{segfault@jabber.ru}
\homepage{http://www.haskellers.com/user/AlekseyUymanov}
\title{A Haskeller}
%% \quote{a Haskell programmer}                                 % optional, remove / comment the line if not wanted


\newcommand*{\hlink}[1]{\textcolor{blue}{\texttt{\underline{\href{#1}{#1}}}}}

\begin{document}
%\begin{CJK*}{UTF8}{gbsn}                          % to typeset your resume in Chinese using CJK
%-----       resume       ---------------------------------------------------------
\makecvtitle

\section{Personal}
\cvitem{Age}{27}
\cvitem{Social Status}{Single}

\section{Education}
\cventry{2007--2009}{Master}{Tyumen Oil and Gas
  Institute}{Tyumen}{\textit{4.8/5}}{Department of Applied Mechanics}

\cventry{2003--2007}{Bachelor}{Tyumen Oil and Gas
  Institute}{Tyumen}{\textit{4.7/5}}{Department of Applied Mechanics}

\section{Experience}
\subsection{Vocational}

\cventry{2013 (current)}{Embedded system
  programmer}{CyberStep}{Tyumen}{}{Developing embedded program for plasma
  cutting machine}

\cventry{2012--2013}{Web programmer}{Iris Integration}{Tyumen}{}{Developing web
  sites on PHP, Python, Ruby:%
  \begin{itemize}
  \item \hlink{http://qrgorod.com}: project on PHP and Mediawiki
  \item \hlink{http://qrgolos.com}: project on Ruby on Rails
  \end{itemize}}

\cventry{2009--2012}{System administrator}{SAU TO MIAC}{Tyumen}{}{}

\cventry{2007--2009}{Design Engineer}{Jsc Sibneftemash}{Tyumen}{}{Calculation of
  the design of process tanks}

\cventry{2006--2007}{System administrator}{MUPGT "Tyumengortrans"}{Tyumen}{}{}

\subsection{Projects}

\cventry {2013 (currently working on)}{Haskell}{HDBC}{\hlink{https://github.com/s9gf4ult/hdbc}}{}
         {Developing new flexible HDBC with some set of features and fixed architecture issues.}

\cventry {2013 (currently working on)}{Haskell}{HDBC-postgresql}{\hlink{https://github.com/s9gf4ult/hdbc-postgresql}}{}
         {The fork of original HDBC driver to work with new HDBC.}

\cventry
    {2013}{Haskell}{quickcheck-assertions}%
    {\hlink{https://github.com/s9gf4ult/quickcheck-assertions}}{}{%
      Package for property-based testing providing pretty-printed test results.}
    
\cventry{2013}{Haskell}{Convertible}
        {\hlink{https://github.com/s9gf4ult/convertible}}{}
        {Fork of the original Convertible, patches are not accepted for now.}

\cventry{2013}{Haskell}{Haskell-Decimal}
        {\hlink{https://github.com/PaulJohnson/Haskell-Decimal}}{}
        {Improved testing, found and fixed multiplication bug, added instances
          to make posible to input literal Decimals.}

\cventry{2013}{Haskell}{Persistent}%
        {\hlink{https://github.com/yesodweb/persistent}}{}{%
          There is no actually usable changes, just some instances to support
          new types in persistent fields.}
    
\cventry{2011--2012}{Python}{Track Deal}%
        {\hlink{https://github.com/s9gf4ult/track-deal}}{}
        {Python program for tracking stock market operations. It is created for
          myself because there was no good Open Source trader log program.}

\section{Computer skills}
\cvitem{Haskell}{%
  Solid Haskell skills:%
  \begin{itemize}
  \item Advanced type-level programming (GADTs, TypeFamilies). While hacking
    Persistent and HDBC
  \item Property-based testing with QuickCheck.
  \item Monadic parser combinators (parsec, attoparsec)
  \item Streaming data processing (Conduits)
  \item Resource management (ResourceT)
  \item Monadic EDSLs
  \item FFI
  \end{itemize}}
\cvitem{Linux}{%
  Experienced Linux user and administrator (Linux is my main home and work OS):
  \begin{itemize}
  \item Virtualisation (kvm, libvirt)
  \item Networking (iptables, routing, traffic shaping)
  \item Data storage (LVM, RAID, iSCSI)
  \end{itemize}}
\cvitem{Databases}{%
  \begin{itemize}
  \item Solid SQL skills
  \item Good knowledge of low-level C interface (especially
    PostgreSQL-libpq and SQLite)
  \item NoSql skills (MongoDB, Memcached, Redis)
  \end{itemize}}%
\cvitem{Python}{%
  Solid Python skills. Good knowledge of PyGtk and Django}
\cvitem{Ruby}{%
  Good skills with Ruby on Rails and and Ruby infrastructure}
\cvitem{Others}{%
  \begin{itemize}
  \item Git, SVN, Hg
  \item Emacs
  \item \LaTeX
  \end{itemize}}

\section{Languages}
\cvitemwithcomment{English}{Beginner}{Good reading and writing, low verbal skills}
\cvitemwithcomment{Russian}{Native}{}

\nocite{*}
\end{document}
